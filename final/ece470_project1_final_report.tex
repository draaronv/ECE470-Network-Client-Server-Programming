\documentclass[11pt]{article}
\usepackage{amsmath}
\usepackage{pdfpages}
\usepackage{microtype}
\usepackage{algorithm}
\usepackage[noend]{algpseudocode}
\usepackage{setspace}
\usepackage{mathtools}
\DeclarePairedDelimiter\ceil{\lceil}{\rceil}
\DeclarePairedDelimiter\floor{\lfloor}{\rfloor}
\usepackage{hyperref}
\usepackage{listings}
\usepackage{lstautogobble}
\usepackage{float}
\usepackage{amssymb}
\usepackage{physics}
\usepackage[margin=0.75in]{geometry}
\usepackage{multicol}
\usepackage{titlesec}
\usepackage{titling}
\usepackage{graphicx}
\usepackage{xcolor}
\definecolor{codegreen}{rgb}{0,0.6,0}
\definecolor{codegray}{rgb}{0.5,0.5,0.5}
\definecolor{codepurple}{rgb}{0.58,0,0.82}
\definecolor{backcolour}{rgb}{0.95,0.95,0.92}
\lstset{language=C++,
	backgroundcolor=\color{backcolour},   
	commentstyle=\color{codegreen},
	keywordstyle=\color{magenta},
	numberstyle=\tiny\color{codegray},
	stringstyle=\color{codepurple},
	basicstyle=\scriptsize\ttfamily,
	commentstyle=\ttfamily\itshape\color{gray},
	stringstyle=\ttfamily,
	showstringspaces=false,
	breaklines=true,
	autogobble=true
}
\setcounter{tocdepth}{3}
\usepackage[nottoc,notlot,notlof]{tocbibind}
\title{\Huge{ECE470:Network Client-Sever Programming}\\\LARGE{Project 1: Part 1}}
\author{Aaron Jesus Valdes}
\begin{document}
	\maketitle
	\clearpage
	\tableofcontents
	\onehalfspacing
	\clearpage
	\section{Project Description}
	Designing a smart home remote access system which follows the Network/Client architecture. This system allows the user to control certain devices from their home such lights, alarm, and locks from a remote location.
	\section{Specifications}
		\begin{itemize}
			\item Supports smart lights
				\begin{itemize}
					\item Can turn on and off lights
					\item Can change the color and brightness of lights
					\item Groups lights into rooms
				\end{itemize}
			\item Supports alarm system
				\begin{itemize}
					\item Can arm and disarm the alarm
					\item Can have more than one alarm system
					\item Can store 4 digit pin per alarm 
				\end{itemize}
			\item Supports Electronic Locks
				\begin{itemize}
					\item Can open and lock the locks
					\item Can have more than 4 locks
					\item Can store a 5 digit pin per lock
				\end{itemize}
			\item Supports more devices
				\begin{itemize}
					\item Can control the thermostat
						\subitem Turn on or off heat/cold
						\subitem Set temperature
					\item Can check the security cameras
						\subitem Can store the video from a security camera
					\item Can control blinds and drapes
						\subitem Opens and closes the blinds
				\end{itemize}
		\end{itemize}
	\section{Assumptions}
		\begin{itemize}
			\item We are only working with a single house
			\item There are multiple rooms in the house
			\item Each room has multiple lights
			\item There is only one alarm
			\item There are at least 4 locks
			\item One user at a time can connect
			\item The server is in the home
			\item The server can be accessed both inside and outside of the house
			\item The user needs to check and alter status of the devices
			\item The method of communication is
			\item The server must be protected by having a login page
		\end{itemize}
	\section{Initial Design}
		\subsection*{Data Model}
			\begin{figure}[H]
				\centering
				\includegraphics[scale=0.5]{data_model}
				\caption{Data models follow the specifications}
				\label{fig:datamodel}
			\end{figure}
		This is the simplest data model which we can use as a template given the fact that as we move one with this project we are going to be adding more features such as setting RGB colors,change lock combinations, setting the brightness, and later on be able to add more devices
		\subsection{Business Logic}
			\subsubsection{Basic Operations}
				\begin{itemize}
					\item Login/Logout
					\item Check Status of each device
					\item Change Status of each devices
				\end{itemize}
			\subsubsection{Storyboard}
				\begin{figure}[H]
					\centering
					\includegraphics[scale=0.5]{StoryBoard}
					\caption{}
					\label{fig:storyboard}
				\end{figure}
			The reason I chose to divide my check status and change status  based on devices is because that is the method that my google home uses show the status of my devices and change the status of my devices. In my opinion the most effective method is by simply having a single main menu with a all the options since it leads to less amount of packets being transferred back and forth between the client and the server and can therefore reduce bandwidth.
			As we move on, I am going to implement more menus for the alarm,locks, and lights as way to provide more features for each device
		\subsection{Sever state chart}
			\begin{figure}[H]
				\centering
				\includegraphics[scale=0.5]{server_state_diagram}
				\label{fig:serverstatediagram}
			\end{figure}
		The server state diagram starts by simply waiting for the client to establish a connection. After establishing a connection, the user is going to be asked to send their username and password which is going to be check by the server based on the data model. If the username and password is correct then we move on to the main menu else we return back to the server login menu. After entering the main menu, the user needs to send the server a request for either checking status of a device, changing the status of a device, or logging out . After the user request the either check or change the status, they are going to receive the list of devices available to choose from. After requesting a device, the user is going to either the rooms, locks, or alarms available to them so they can either change or check the status. For changing the status there is an extra step in which the server updates the data model for each change and returns back to displaying the status of that device. For each menu there is always a return option which is simply by requesting to press enter. 
		\subsection{Client State Chart}
			\begin{figure}[H]
				\centering
				\includegraphics[scale=0.5]{Client_State_Diagram}
				\label{fig:clientstatediagram}
			\end{figure}
		The client state diagram starts by simply establishing a connection between the server and the client. Then the server sends the client a message,which the user uses to make a request. After the request is sent to the server, the server makes a decision based on that request and executes a command. The connection between the server and the client ends when the user decides to log out.
		\subsection{Application Protocol Design}
			\begin{table}[H]
				\centering
				\resizebox{\textwidth}{!}{%
					\begin{tabular}{llll}
						\hline
						\multicolumn{1}{|l|}{Client Application Protocols} & \multicolumn{1}{l|}{Parameters}                 & \multicolumn{1}{l|}{Body}    & \multicolumn{1}{l|}{Description}                                                                            \\ \hline
						\multicolumn{1}{|l|}{START}                        & \multicolumn{1}{l|}{none}                       & \multicolumn{1}{l|}{none}    & \multicolumn{1}{l|}{Establishes connection with the server}                                                 \\ \hline
						\multicolumn{1}{|l|}{CHOICE}                       & \multicolumn{1}{l|}{Lines=24 character,lines=0} & \multicolumn{1}{l|}{None}    & \multicolumn{1}{l|}{Based on the options of the server, the user can send up to 24 character to the server} \\ \hline
						\multicolumn{1}{|l|}{END}                          & \multicolumn{1}{l|}{None}                       & \multicolumn{1}{l|}{none}    & \multicolumn{1}{l|}{Ends the connection between the client and the server}                                  \\ \hline
						&                                                 &                              &                                                                                                             \\
						&                                                 &                              &                                                                                                             \\
						&                                                 &                              &                                                                                                             \\ \hline
						\multicolumn{1}{|l|}{Server Application Protocols} & \multicolumn{1}{l|}{Parameters}                 & \multicolumn{1}{l|}{Body}    & \multicolumn{1}{l|}{Description}                                                                            \\ \hline
						\multicolumn{1}{|l|}{USER}                         & \multicolumn{1}{l|}{Lines=24 character}         & \multicolumn{1}{l|}{Line}    & \multicolumn{1}{l|}{Ask the user for the username}                                                          \\ \hline
						\multicolumn{1}{|l|}{PASS}                         & \multicolumn{1}{l|}{Line=24 character}          & \multicolumn{1}{l|}{Line}    & \multicolumn{1}{l|}{Ask the user for the password}                                                          \\ \hline
						\multicolumn{1}{|l|}{LIST}                         & \multicolumn{1}{l|}{Line=24 character}          & \multicolumn{1}{l|}{Line(s)} & \multicolumn{1}{l|}{Gives the user options which they can chose from}                                       \\ \hline
						\multicolumn{1}{|l|}{ERROR}                        & \multicolumn{1}{l|}{Line=24 character}          & \multicolumn{1}{l|}{Line(s)} & \multicolumn{1}{l|}{Send the user information about an error in the system and waits for his choice}        \\ \hline
						\multicolumn{1}{|l|}{END}                          & \multicolumn{1}{l|}{None}                       & \multicolumn{1}{l|}{None}    & \multicolumn{1}{l|}{The server ends the connection between the client and the server}                       \\ \hline
					\end{tabular}
				}
			\end{table}
	\section{Example}
		\begin{figure}[H]
			\centering
			\includegraphics[scale=0.5]{Examples}
			\label{fig:examples}
		\end{figure}
	\section{Data model}
			\begin{lstlisting}[language=C++]
				#ifndef _DATAMODEL_H_
				#define _DATAMODEL_H
				#include <iostream>
				#include <string>
				#include <vector>
				using namespace std;
				
				class light
				{
				public:
				string name;
				bool status;
				};
				
				
				
				class room
				{
				public:
				vector<light> l;
				};
				
				
				
				class rooms
				{
				public:
				vector<room> r;
				string name;
				int light_number;
				};
				
				
				class alarm
				{
				public:
				string name;
				int code;
				};
				
				
				
				class lock
				{
				public:
				string name;
				int code;
				};
				
				
				
				class locks
				{
				public:
				vector<lock> llock;
				};
				
				
				class home
				{
				private:
				string password;
				string username;
				public:
				rooms n;
				alarm a;
				locks lo;
				void setUsername(string newUsername);
				void setPassword(string password);
				};
				#endif
			\end{lstlisting}
		\section{Message Protocol}
			\subsection{Client Marshal}
				\subsubsection{Header file}
					\begin{lstlisting}[language=C++]
						#ifndef _CLIENT_MARSHAL_H_
						#define _CLIENT_MARSHAL_H_
						#include <iostream>
						#include <string>
						#include <stdio.h>
						#include <stdlib.h>
						#include <sstream>
						#include <vector>
						using namespace std;
						class server_message
						{
						public:
						string options;
						string message;
						vector<string> lines;
						int num_lines;
						void printing()
						{
						cout<<options<<"    "<<num_lines<<"     "<<message<<endl;
						for(unsigned int i=0;i<lines.size();i++)
						{
						cout<<lines[i]<<endl;
						}
						
						}
						};
						
						class client_message
						{
						public:
						string options;
						int decision;
						void printing()
						{
						cout<<options<<"    "<<decision;
						}
						};
						
						server_message unmarshal(string message)
						{
						string command;
						server_message res;
						stringstream ss(message);
						ss>>command;
						if(command=="USER")
						{   res.options="USER";
						getline(ss,res.message);
						}
						else if(command=="PASS")
						{
						res.options="PASS";
						getline(ss,res.message);
						}
						else if(command=="LIST")
						{
						res.options="LIST";
						getline(ss,res.message,'\\');
						ss>>res.num_lines;
						vector<stringstream> tt(res.num_lines);
						string temp;
						int i=0;
						while(ss>>temp)
						{
						if(res.message=="\\\\")
						{
						break;
						}
						else if(temp=="\\")
						{
						i++;
						}
						else
						{
						tt[i]<<temp<<" ";
						}
						}
						for(unsigned int j=0;j<tt.size();j++)
						{
						string temp2;
						getline(tt[j],temp2);
						res.lines.push_back(temp2);
						}
						}
						else if(command=="ERROR")
						{
						res.options="ERROR";
						getline(ss,res.message);
						ss>>res.message;
						}
						else if(command=="END")
						{
						res.options="END";
						getline(ss,res.message);
						ss>>res.message;
						}
						else
						{
						}
						return res;
						}
						
						
						string marshal(client_message cm)
						{
						stringstream ss;
						string result;
						if(cm.options=="START")
						{
						ss<<cm.options<<" "<<cm.decision;
						}
						else if(cm.options=="CHOICE")
						{
						ss<<cm.options<<" "<<cm.decision;
						}
						else if(cm.options=="END")
						{
						ss<<cm.options<<" "<<cm.decision;
						}
						else if(cm.options=="ERROR")
						{
						ss<<cm.options<<" "<<cm.decision;
						}
						else
						{
						} 
						getline(ss,result);
						return result;
						}
						#endif
					\end{lstlisting}
				\subsubsection{Test File}
					\begin{lstlisting}[language=C++]
						#include <iostream>
						#include <string>
						#include <stdio.h>
						#include <stdlib.h>
						#include <sstream>
						#include "client_marshal.h"
						using namespace std;
						int main(int argc,char*argv[])
						{
						if(argc==2)
						{
						cout<<"Testing unmarshal"<<endl;
						server_message test1;
						test1 =unmarshal(argv[1]);
						test1.printing();
						}
						
						if(argc>2)
						{
						cout<<"Testing Marshall"<<endl;
						client_message test;
						test.decision=atoi(argv[2]);
						test.options=argv[1];
						cout<<marshal(test)<<endl;
						}
						return 0;
						}
					\end{lstlisting}
				\subsubsection{Makefile}
					\begin{lstlisting}[language=C++]
						all:client server_marshal.h client_marshal.h
						echo Testing the client
						client:client.cpp client_marshal.h
						g++ client.cpp -o client				
						test_client: client client.cpp client_marshal.h
						@./client "USER Enter Your Username:"
						@./client "PASS Enter your Password:"
						@./client "ERROR There is an Error:"
						@./client "END Ending the Connection"
						@./client "LIST Here are the options: \ 1 1.Option 1 \\"
						@./client "LIST Here are the options: \ 2 1.Option 1 \ 2. Option 2 \\"
						@./client "LIST Here are the options: \ 3 1.Option 1 \ 2. Option 2 \ 3. Option 3 \\"
						@./client "LIST Here are the options: \ 4 1.Option 1 \ 2. Option 2 \ 3. Option 3 \ 4. Option 4 \\"
						@ echo
						@ echo
						@./client "START" "1"
						@./client "CHOICE" "1"
						@./client "END" "1"
						@./client "ERROR" "1"
						clean:
						rm client
					\end{lstlisting}
				\subsection{Results}	
					\begin{figure}[H]
						\centering
						\includegraphics[scale=0.5]{"~/Pictures/Screenshot from 2021-03-02 21-02-45"}
						\caption{Results for the client marshalling}
						\label{fig:screenshot-from-2021-03-02-21-02-45}
					\end{figure}
			\subsection{Server Marshal}
				\subsubsection{Header File}
					\begin{lstlisting}[language=C++]
						#ifndef _SERVER_MARSHAL_H_
						#define _SERVER_MARSHAL_H
						#include <iostream>
						#include <string>
						#include <stdio.h>
						#include <stdlib.h>
						#include <sstream>
						#include <vector>
						using namespace std;
						class client_message
						{
						public:
						string command;
						int decision;
						void printing()
						{
						cout<<command<<"    "<<decision<<endl;
						}
						};
						
						class server_message
						{
						public:
						string command;
						string messa;
						vector<string> lines;
						int num_lines;
						};
		
						client_message unmarshal(string message)
						{
						string command;
						int decision;
						client_message res;
						istringstream ss(message);
						ss>>command>>decision;
						res.command=command;
						res.decision=decision;
						return res;
						}
						
						string marshal(server_message sm)
						{
						stringstream ss;
						string result;
						string c[5]={"USER","PASS","LIST","ERROR","END"};
						if(sm.command == c[0])
						{
						ss<<sm.command<<" "<<sm.messa<<"\\";
						}
						else if(sm.command == c[1])
						{
						ss<<sm.command<<" "<<sm.messa<<"\\";
						}
						else if(sm.command == c[2])
						{
						ss<<sm.command<<" "<<sm.num_lines<<" "<<sm.messa<<"\\";
						for(int i=0;i<sm.num_lines;i++)
						{
						ss<<sm.lines[i]<<"\\";
						}
						ss<<"\\\\";
						}
						else if(sm.command == c[3])
						{
						ss<<sm.command<<" "<<sm.messa;
						}
						else if(sm.command == c[4])
						{
						ss<<sm.command<<" "<<sm.messa;
						}
						else
						{
						}
						getline(ss,result);
						return result;
						}
						#endif
					\end{lstlisting}
				\subsubsection{Test program}
					\begin{lstlisting}[language=C++]
						#include "server_marshal.h"
						using namespace std;
						
						int main(int argc,char*argv[])
						{
						if(argc<3)
						{
						cout<<"Testing unmarshal"<<endl;
						client_message test=unmarshal(argv[1]);
						test.printing();
						}
						else
						{
						cout<<"Testing Marshal"<<endl;
						server_message test1;
						test1.command=argv[1];
						test1.messa=argv[2];
						switch(argc)
						{
						case 5:
						{
						test1.num_lines=atoi(argv[3]);
						test1.lines.push_back(argv[4]);
						break;
						}
						case 6:
						{
						test1.num_lines=atoi(argv[3]);
						test1.lines.push_back(argv[4]);
						test1.lines.push_back(argv[5]);
						break;
						}
						case 7:
						{
						test1.num_lines=atoi(argv[3]);
						test1.lines.push_back(argv[4]);
						test1.lines.push_back(argv[5]);
						test1.lines.push_back(argv[6]);
						break;
						}
						}
						cout<<marshal(test1)<<endl<<endl;
						}
						return 0;
						}
					\end{lstlisting}
				\subsubsection{Makefile}
					\begin{lstlisting}[language=C++]
						all:server  server_marshal.h
						echo Testing the client
						
						server:server.cpp server_marshal.h
						g++ server.cpp -o server
						
						test_server: server server.cpp server_marshal.h
						@./server "START 2"
						@./server "CHOICE 5"
						@./server "END 1"
						@./server "ERROR 2"
						@./server "USER" " Enter your Username: "
						@./server "PASS" "Enter your password: "
						@./server "ERROR" "You have an error "
						@./server "END" "Ending Connection"
						@./server "LIST" "This are the options" "3" " 1. Option 1 " " 2. Option 2 " " 3. Option 3 "
						@./server "LIST" "This are the options" "2" " 1. Option 1 " " 2. Option 2 "
						@./server "LIST" "This are the options" "1" " 1. Option 1 "
						clean:
						rm server
					\end{lstlisting}
				\subsubsection{Result}
					\begin{figure}[H]
						\centering
						\includegraphics[scale=0.4]{"~/Pictures/Screenshot from 2021-03-02 21-20-31"}
						\caption{Results for my message protocol}
						\label{fig:screenshot-from-2021-03-02-21-20-31}
					\end{figure}
	\section{TCP Protocols}
		\subsection{Client}
			\subsubsection{Header File}
				\begin{lstlisting}[language=C++]
					#ifndef _TCP_CLIENT_H_
					#define _TCP_CLIENT_H_
					#include <iostream>
					#include <string>
					#include<stdio.h>
					#include<sys/types.h>
					#include<sys/socket.h>
					#include<netinet/in.h>
					#include<arpa/inet.h>
					#include<netdb.h>
					#include<string.h>
					#include<stdlib.h>
					#include<unistd.h>
					#define BACKLOG 5
					#define DE_SERVER_PORT 12345
					#define DE_CLIENT_PORT 12345
					#define MAXBUFFERLEN 1024
					#define SERV_IP "127.0.0.1"
					using namespace std;
					
					class client
					{
					private:
					int sockfd_server;
					int connection;
					int sockfd_client;
					int sockfd_client_acc;
					string server_route;
					string client_route;
					int server_port;
					int client_port;
					sockaddr_in server_address={0};
					sockaddr_in client_address={0};
					public:
					client();
					client(string serverIP,int serverPort);
					int send_message(string message);
					string receive_message();
					void closing_all();
					void close_single(int option);
					};
					
					client::client()
					{
					connection=1;
					sockfd_client_acc=0;
					server_route=SERV_IP;
					sockfd_server=socket(AF_INET,SOCK_STREAM,0);
					sockfd_client=socket(AF_INET,SOCK_STREAM,0);
					if(sockfd_server==-1)
					{
					perror("Failed to create socket for Communicating with the server");
					exit(EXIT_FAILURE);
					}
					
					if(sockfd_client==-1)
					{
					perror("Failed to create socket for listening for message from Server");
					exit(EXIT_FAILURE);
					}
					//To send messages
					server_address.sin_family=AF_INET;
					server_address.sin_port=htons(DE_SERVER_PORT);
					inet_pton(AF_INET,server_route.c_str(),&(server_address.sin_addr));
					memset(&(server_address.sin_zero),'\0',8);
					
					//To make the client listen but especially to bind it to a port
					client_address.sin_family=AF_INET;
					client_address.sin_port=ntohs(DE_CLIENT_PORT);
					client_address.sin_addr.s_addr=INADDR_ANY;
					memset(&(client_address.sin_zero),'\0',8);
					
					}
					
					client::client(string serverIP,int port)
					{
					connection=1;
					sockfd_client_acc=0;
					server_route=serverIP;
					sockfd_server=socket(AF_INET,SOCK_STREAM,0);
					sockfd_client=socket(AF_INET,SOCK_STREAM,0);
					if(sockfd_server==-1)
					{
					perror("Failed to create socket for Communicating with the server");
					exit(EXIT_FAILURE);
					}
					
					if(sockfd_client==-1)
					{
					perror("Failed to create socket for listening for message from Server");
					exit(EXIT_FAILURE);
					}
					//To send messages
					server_address.sin_family=AF_INET;
					server_address.sin_port=htons(port);
					inet_pton(AF_INET,server_route.c_str(),&(server_address.sin_addr));
					memset(&(server_address.sin_zero),'\0',8);
					
					//To make the client listen but especially to bind it to a port
					client_address.sin_family=AF_INET;
					client_address.sin_port=ntohs(port);
					client_address.sin_addr.s_addr=INADDR_ANY;
					memset(&(client_address.sin_zero),'\0',8);
					}
					
					
					
					
					int client::send_message(string message)
					{
					char buffer[MAXBUFFERLEN];
					FILE *stream;
					//Converting string to char[]
					int tn=message.size();
					char temp[tn+1];
					strcpy(temp,message.c_str());
					//Creating the stream for our message
					stream=fmemopen(temp,strlen(temp),"r");
					int num_char_read=fread(buffer+1,sizeof(char),sizeof(buffer),stream);
					buffer[0]=num_char_read;
					sockfd_server=socket(AF_INET,SOCK_STREAM,0);
					connection=connect(sockfd_server,(const struct sockaddr*)&server_address,sizeof(struct sockaddr));
					if(connection==-1)
					{
					perror("Failure to connect to server");
					exit(EXIT_FAILURE);
					close(sockfd_server);
					return -1;
					}
					int sending=write(sockfd_server,(const char*)buffer,strlen(buffer)+1);
					if(sending==-1)
					{
					perror("Failure to send package");
					exit(EXIT_FAILURE);
					close(sockfd_server);
					return -1;
					}
					close(sockfd_server);
					return 1;
					}
					
					string client::receive_message()
					{
					FILE *stream;
					char *bp;
					size_t size;
					stream=open_memstream(&bp,&size);
					char buffer[MAXBUFFERLEN];
					
					if(sockfd_client_acc==0)
					{
					int rc=bind(sockfd_client,(const struct sockaddr*)&client_address,sizeof(client_address));
					if(rc==-1)
					{
					perror("Failed to bind client_address to socket");
					close(sockfd_client);
					exit(EXIT_FAILURE);
					}
					int listening=listen(sockfd_client,BACKLOG);
					if(listening==-1)
					{
					perror("Failed to listen to socket");
					close(sockfd_client);
					exit(EXIT_FAILURE);
					}
					}
					
					unsigned int sin_size=sizeof(struct sockaddr_in);
					sockfd_client_acc=accept(sockfd_client,(struct sockaddr *)&server_address,&sin_size);
					int received=read(sockfd_client_acc,(char *)buffer,MAXBUFFERLEN);
					if(received==-1)
					{
					perror("Failed to received package");
					close(sockfd_client);
					close(sockfd_client_acc);
					exit(EXIT_FAILURE);
					}
					fwrite(buffer+1,sizeof(char),buffer[0],stream);
					fflush(stream);
					string temp(bp);
					return temp;
					}
					
					void client::closing_all()
					{
					close(sockfd_server);
					close(sockfd_client);
					close(sockfd_client_acc);
					}
					
					void client::close_single(int option)
					{
					if(option==0)
					{
					close(sockfd_server);
					}
					else if(option==2)
					{
					close(sockfd_client);
					}
					else if(option==3)
					{
					close(sockfd_client_acc);
					}
					else
					{
					close(sockfd_server);
					close(sockfd_client);
					close(sockfd_client_acc);
					}
					}
					#endif
				\end{lstlisting}
			\subsubsection{Test Program}
				Test to make sure the client can send a message
				\begin{lstlisting}[language=C++]
					\#include "tcp\_client.h"
					using namespace std;
					int main()
					\{
					client test;
					test.send\_message("START 2");
					test.send\_message("CHOICE 5");
					test.send\_message("END 1");
					test.send\_message("ERROR 2");
					test.send\_message("end");
					test.closing\_all();
					return 0;
					\}
				\end{lstlisting}
				Test to make sure that the server can receive a message and then send that message back
				\begin{lstlisting}[language=C++]
					#include "tcp_client.h"
					using namespace std;
					
					int main()
					{
					client test;
					client test2("127.0.0.1",5555);
					string temp="follow";
					while (temp!="end")
					{
					temp=test.receive_message();
					cout<<temp<<endl;
					sleep(1);
					test2.send_message(temp);
					}
					test.closing_all();
					return 0;
					}
				\end{lstlisting}
				I used the sleep command because both the server and the client are running synchrounously which mean that when the server tries to send back the message there is no one listening.\\
				We can simply just run this two functions on two threads which is simpler.
			\subsubsection{Makefile}
				\begin{lstlisting}[language=C++]
					all:client server
					client:tcp\_client.h testing\_client.cpp
					g++ -g testing\_client.cpp -o client
					server:tcp\_client.h testing\_server.cpp
					g++ -g testing\_server.cpp -o server
					test: all
					./server >test.txt | ./client
					clean:
					rm client server
				\end{lstlisting}
			\subsubsection{Results}
				\begin{figure}[H]
					\centering
					\includegraphics[scale=0.5]{"~/Pictures/Screenshot from 2021-03-02 22-17-39"}
					\caption{Results on the server side of the message from the client}
					\label{fig:screenshot-from-2021-03-02-22-17-39}
				\end{figure}
				\begin{figure}[H]
					\centering
					\includegraphics[scale=0.5]{"~/Pictures/Screenshot from 2021-03-02 22-59-09"}
					\caption{Results of the server side of the message send from the client which is then sent back to the client }
					\label{fig:screenshot-from-2021-03-02-22-59-09}
				\end{figure}	
		\subsection{Client}
			\subsubsection{Header File}
				Is the same as the server
			\subsubsection{Test Program}
				This program simply test if the client can send messages
				\begin{lstlisting}[language=C++]
					\#include "tcp\_client.h"
					using namespace std;
					int main()
					\{
					client test;
					string temp="follow";
					while (temp!="end")
					\{
					temp=test.receive\_message();
					cout<<temp<<endl;
					\}
					test.closing\_all();
					return 0;
					\}
				\end{lstlisting}
				This program simply test that I can send a message to the server and then the server will return it back
				\begin{lstlisting}[language=C++]
					#include "tcp_client.h"
					using namespace std;
					
					int main()
					{
					client test;
					client test2("127.0.0.1",5555);
					string temp;
					test.send_message("START 2");
					temp=test2.receive_message();
					cout<<temp<<endl;
					
					test.send_message("CHOICE 5");
					temp=test2.receive_message();
					cout<<temp<<endl;
					
					test.send_message("END 1");
					temp=test2.receive_message();
					cout<<temp<<endl;
					
					test.send_message("ERROR 2");
					temp=test2.receive_message();
					cout<<temp<<endl;
					
					test.send_message("end");
					temp=test2.receive_message();
					cout<<temp<<endl;
					
					test.closing_all();
					return 0;
					}
				\end{lstlisting}
			\subsubsection{Makefile}
				\begin{lstlisting}[language=C++]
					all:client server
					client:tcp\_client.h testing\_client.cpp
					g++ -g testing\_client.cpp -o client
					server:tcp\_client.h testing\_server.cpp
					g++ -g testing\_server.cpp -o server
					test: all
					./server >test.txt | ./client
					clean:
					rm client server
				\end{lstlisting}
			\subsubsection{Results}
				\begin{figure}[H]
					\centering
					\includegraphics[scale=0.5]{"~/Pictures/Screenshot from 2021-03-02 22-17-39"}
					\caption{Results on the server side of the message from the client}
					\label{fig:screenshot-from-2021-03-02-22-17-39}
				\end{figure}
				\begin{figure}[H]
					\centering
					\includegraphics[scale=0.5]{"~/Pictures/Screenshot from 2021-03-02 22-58-51"}
					\caption{Results of the message send to the server from the client which was sent back to to the client}
					\label{fig:screenshot-from-2021-03-02-22-58-51}
				\end{figure}
\end{document}